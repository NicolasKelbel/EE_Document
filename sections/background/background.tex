\section*{Background}
\label{sec:background}

\subsection*{Fundamentals of Fluid Dynamics}
Firstly, the definition of vorticity, a vortex and a bluff body must be understood. Vorticity is the 6“curl of velocity”, in other words the 7“rate of local fluid rotation”. A vortex is the 7“rotating motion of a fluid around a common centerline”, characterized by the “vorticity in the fluid”. Bluff body is defined as an object that, 8“as a result of its shape, has separated flow over a substantial part of its surface”. 

Next, one must distinguish between a Newtonian and a non-Newtonian fluid. A Newtonian fluid is a fluid in which 3“viscosity is constant and not dependent on the shear rate”, whereas a non-Newtonian is a fluid in which its viscosity is dependent on the shear and therefore is not constant. 

Moreover, one must also consider the type of flow. There are two key factors of flow when investigating vortex shedding: laminar vs. turbulent flow and compressible vs. incompressible flow. Laminar flow is characterized as a 2“flow condition in which fluid particles follow smooth and steady streamlines with little movement of particles between adjacent layers”. On the other hand, turbulent flow is known to have “chaotic variations in the magnitude and direction of fluid particle velocity and amplitude of pressure”. A compressible flow is a flow in which the 3“density of a fluid does not remain constant”, conversely, an incompressible flow is a flow in which the “density of fluid remains constant”. To adhere to the scope of this essay, vortex shedding in a Newtonian fluid which exhibits laminar and incompressible flow will be analyzed.

Lastly, two dimensionless numbers must be considered: the Reynolds’ and Strouhal number. The Reynolds’ number expresses the 4“ratio between inertial and viscous forces”. Given by the equation (equation), it allows one to classify if a flow is laminar or turbulent, with low Reynolds’ numbers signifying laminar flow and high Reynolds’ numbers indicating turbulent flow. The Strouhal number denotes the 5“ratio of inertial forces due to the local acceleration of the flow to the inertial forces due to the convective acceleration”. 5Given by the equation (equation), high Strouhal numbers represent a flow in which oscillations dominate, whereas low Strouhal numbers characterize a flow in which oscillations are carried away by the high-velocity fluid.


