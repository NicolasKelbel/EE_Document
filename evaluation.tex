\section{Evaluation}

\subsection{Evaluation of the Theoretical Investigation}
The simulation used for the theoretical investigation was based off of a tutorial case made by a professional, with minor changes made in order to adhere to the aim of this essay. As detailed in \Cref{sec:theoreticalMethod}, the simulation was run five times for each bluff body, yielding identical results for each repetition and therefore providing a strong foundation for the results of this investigation. Unlike the practical investigation, the simulation provided a method of study in which a two-dimensional flow could be investigated without the impact of boundary walls. Moreover, the simulated environment enabled strict control over the constant variables, thereby minimizing the influence of confounding factors, ensuring the alteration of vortex shedding frequency could be solely attributed to the change in the number of streamwise faces. 

The measurement of the lift coefficient $C_L$ was done within the simulation, eliminating the uncertainties introduced by external measurement equipment. Furthermore, the determination of the time period was completed for all peaks and troughs, with an average providing the final time period, therefore reducing the impact of anomalies on the results.

Nevertheless, as mentioned in \Cref{sec:conclusion}, numerical uncertainties introduced by numerous factors such as the mesh resolution may remain a reason for error, especially due to the difficult nature of quantifying these potentially mutually canceling factors \parencite{city7565}. Computational limitations restricted both the unit size of the mesh of the domain and the time step value (\verb*|deltaT|), therefore possibly introducing discrepancies between the modeled and actual fluid behavior.

This was partially solved by the inclusion of a BOI, decreasing the unit size of the mesh around the bluff body and in the wake region. Coupled with a decreased unit size at the edge of the bluff body, the interaction of the bluff body with the flow was simulated more accurately. 

\subsection{Evaluation of the Practical Investigation}
The inability to achieve laminar flow rendered the practical investigation nonviable for quantitative comparison with simulated data. It would have been necessary to decrease the characteristic length of the bluff bodies, the flow velocity or a combination of both. However, as discussed earlier in \Cref{sec:practicalMethod}, decreasing the size of the bluff bodies would have resulted in smaller vortices, too small to identify by eyesight. In addition, the lack of a less powerful water pump rendered a further decrease in flow velocity \textemdash\ beyond that provided by the regulation valve \textemdash\ unachievable.

Even if a Reynolds number of 100 had been achieved, the presence of the horizontal plate and the surrounding insulation foam walls would have slowed the flow near the surface boundaries, causing boundary layer formation, disrupting the uniform flow. It is fundamentally unfeasible to achieve a truly two-dimensional flow in a three-dimensional physical space \textemdash\ therefore any data would have only served as approximations. 

Additionally, there is an inherent difficulty in identifying the exact point when a vortex is shed through visual inspection. In combination with possible human error, when using the digital stop watch, any data collected would have lacked accuracy.

