\section{Results}
\subsection{Raw Data of the Theoretical Investigation}

\subsubsection{Sample Table of Time vs. Lift Coefficient ($C_L$) for Bluff Body $n=2$}

\begin{table}[H]
	\centering
	\begin{tabular}{|c|c|}
		\hline
		\textbf{Time (s)} & \textbf{$C_L$} \\
		\hline
		$1 \times 10^{-5}$ & $2.0946206 \times 10^{-11}$ \\
		\hline
		$2 \times 10^{-5}$ & $-6.5426717 \times 10^{-12}$ \\
		\hline
		$3 \times 10^{-5}$ & $-7.3241302 \times 10^{-12}$ \\
		\hline
		$4 \times 10^{-5}$ & $-6.6677951 \times 10^{-12}$ \\
		\hline
		$5 \times 10^{-5}$ & $-5.1304516 \times 10^{-12}$ \\
		\hline
		$6 \times 10^{-5}$ & $-4.3635501 \times 10^{-12}$ \\
		\hline
		$7 \times 10^{-5}$ & $-3.6837751 \times 10^{-12}$ \\
		\hline
		$8 \times 10^{-5}$ & $-3.2635206 \times 10^{-12}$ \\
		\hline
		$9 \times 10^{-5}$ & $-2.8660898 \times 10^{-12}$ \\
		\hline
		$1.0 \times 10^{-4}$ & $-2.4321206 \times 10^{-12}$ \\
		\hline
		$1.1 \times 10^{-4}$ & $-2.1246574 \times 10^{-12}$ \\
		\hline
		$1.2 \times 10^{-4}$ & $-1.8807166 \times 10^{-12}$ \\
		\hline
		$1.3 \times 10^{-4}$ & $-1.6461582 \times 10^{-12}$ \\
		\hline
		$1.4 \times 10^{-4}$ & $-1.4100222 \times 10^{-12}$ \\
		\hline
		$1.5 \times 10^{-4}$ & $-1.1994923 \times 10^{-12}$ \\
		\hline
		$1.6 \times 10^{-4}$ & $-1.0306274 \times 10^{-12}$ \\
		\hline
		$1.7 \times 10^{-4}$ & $-8.6746456 \times 10^{-13}$ \\
		\hline
		$1.8 \times 10^{-4}$ & $-7.3730846 \times 10^{-13}$ \\
		\hline
		$1.9 \times 10^{-4}$ & $-6.2413158 \times 10^{-13}$ \\
		\hline
		$2.0 \times 10^{-4}$ & $-5.1195718 \times 10^{-13}$ \\
		\hline
	\end{tabular}
	\label{tab:1FaceClTable}
	\caption{Example of the first 20 values of the table produced by the simulation. Time vs. Lift Coefficient ($C_L$) for bluff body $n=2$.}
\end{table}

\vspace{1em}

\begin{tcolorbox}[width=\textwidth, height=2.3cm, colframe=black, boxrule=0.5pt, sharp corners]
	\vspace{0.6em}
	Only a representative sample is shown here for clarity. Complete datasets available on request. 
\end{tcolorbox}

\clearpage

\subsection{Processed Data of the Theoretical Investigation}

\subsubsection{Representative (Average) Lift Coefficient ($C_L$) Over Time for Each Bluff Body}
\label{sec:C_LvsTime}

\begin{figure}[H]
	\centering
	\includegraphics[width=\textwidth]{images/2face_graph}
	\caption{Average lift coefficient $C_L$ over time for the bluff body $n=2$, based on identical simulation results. The red dashed line indicates the start of periodic vortex shedding at $t = \SI{1}{\second}$.}
	\label{fig:2FaceGraph}
\end{figure}

\begin{figure}[H]
	\centering
	\includegraphics[width=\textwidth]{images/3face_graph}
	\caption{Average lift coefficient $C_L$ over time for the bluff body $n=3$, based on identical simulation results. The red dashed line indicates the start of periodic vortex shedding at $t = \SI{1}{\second}$.}
	\label{fig:3FaceGraph}
\end{figure}

\begin{figure}[H]
	\centering
	\includegraphics[width=\textwidth]{images/4face_graph}
	\caption{Average lift coefficient $C_L$ over time for the bluff body $n=4$, based on identical simulation results. The red dashed line indicates the start of periodic vortex shedding at $t = \SI{1}{\second}$.}
	\label{fig:4FaceGraph}
\end{figure}

\begin{figure}[H]
	\centering
	\includegraphics[width=\textwidth]{images/5face_graph}
	\caption{Average lift coefficient $C_L$ over time for the bluff body $n=5$, based on identical simulation results. The red dashed line indicates the start of periodic vortex shedding at $t = \SI{1}{\second}$.}
	\label{fig:5FaceGraph}
\end{figure}

\begin{figure}[H]
	\centering
	\includegraphics[width=\textwidth]{images/6face_graph}
	\caption{Average lift coefficient $C_L$ over time for the bluff body $n=6$, based on identical simulation results. The red dashed line indicates the start of periodic vortex shedding at $t = \SI{1}{\second}$.}
	\label{fig:6FaceGraph}
\end{figure}

\begin{figure}[H]
	\centering
	\includegraphics[width=\textwidth]{images/7face_graph}
	\caption{Average lift coefficient $C_L$ over time for the bluff body $n=7$, based on identical simulation results. The red dashed line indicates the start of periodic vortex shedding at $t = \SI{1}{\second}$.}
	\label{fig:7FaceGraph}
\end{figure}

\begin{figure}[H]
	\centering
	\includegraphics[width=\textwidth]{images/8face_graph}
	\caption{Average lift coefficient $C_L$ over time for the bluff body $n=8$, based on identical simulation results. The red dashed line indicates the start of periodic vortex shedding at $t = \SI{1}{\second}$.}
	\label{fig:8FaceGraph}
\end{figure}

\begin{figure}[H]
	\centering
	\includegraphics[width=\textwidth]{images/9face_graph}
	\caption{Average lift coefficient $C_L$ over time for the bluff body $n=9$, based on identical simulation results. The red dashed line indicates the start of periodic vortex shedding at $t = \SI{1}{\second}$.}
	\label{fig:9FaceGraph}
\end{figure}

\begin{figure}[H]
	\centering
	\includegraphics[width=\textwidth]{images/10face_graph}
	\caption{Average lift coefficient $C_L$ over time for the bluff body $n=10$, based on identical simulation results. The red dashed line indicates the start of periodic vortex shedding at $t = \SI{1}{\second}$.}
	\label{fig:10FaceGraph}
\end{figure}

\begin{figure}[H]
	\centering
	\includegraphics[width=\textwidth]{images/11face_graph}
	\caption{Average lift coefficient $C_L$ over time for the bluff body $n=11$, based on identical simulation results. The red dashed line indicates the start of periodic vortex shedding at $t = \SI{1}{\second}$.}
	\label{fig:11FaceGraph}
\end{figure}

\begin{figure}[H]
	\centering
	\includegraphics[width=\textwidth]{images/12face_graph}
	\caption{Average lift coefficient $C_L$ over time for the bluff body $n=12$, based on identical simulation results. The red dashed line indicates the start of periodic vortex shedding at $t = \SI{1}{\second}$.}
	\label{fig:12FaceGraph}
\end{figure}


\clearpage


\subsubsection{Sample Calculation of Vortex Shedding Frequency for Bluff Body $n=2$ using Python}

\begin{figure}[H]
	\centering
	\includegraphics[width=\textwidth]{images/2face_graph_sample_Calc}
	\caption{Sample visualization of the peaks and troughs identification on the Lift Coefficient $C_L$ over time graph for bluff body $n=2$. The initial startup phase, colored in gray, is excluded from the detection.}
	\label{fig:2FaceGraphSampleCalc} 
\end{figure}

\begin{tcolorbox}[title=Python Output,fonttitle=\bfseries,
	colframe=black!75!white,colback=gray!10!white,boxrule=0.5pt,
	fontupper=\ttfamily]
	Number of peaks:    65 \\
	Number of troughs:  64 \\
	
	Average time period: 0.06052 s \\
\end{tcolorbox}

Using the outputted average period $ T = \SI{0.06052}{\second} $, the vortex shedding frequency was calculated:

\[
f = \frac{1}{T} = \frac{1}{0.06052} = \SI{16.52346}{\hertz}
\]

This represents the vortex shedding frequency of the bluff body with $ n = 2 $ faces in laminar flow.

\subsubsection{A Comparison of Vortex Shedding Frequency with Increasing $n$ from 2 \textendash\ 12}

\begin{table}[H]
	\centering
	\renewcommand{\arraystretch}{1.3}
	\begin{tabular}{|c|c|}
		\hline
		\textbf{Bluff Body $n$} & \textbf{Vortex Shedding Frequency (\si{\hertz})} \\
		\hline
		$2$  & $16.52346$ \\
		\hline
		$3$  & $16.36393$ \\
		\hline
		$4$  & $16.41497$ \\
		\hline
		$5$  & $16.22323$ \\
		\hline
		$6$  & $16.28664$ \\
		\hline
		$7$  & $16.07976$ \\
		\hline
		$8$  & $16.32387$ \\
		\hline
		$9$  & $16.05910$ \\
		\hline
		$10$ & $16.09528$ \\
		\hline
		$11$ & $16.18909$ \\
		\hline
		$12$ & $16.21797$ \\
		\hline
	\end{tabular}
	\caption{Vortex shedding frequencies for bluff bodies with $n$ streamwise faces.}
	\label{tab:frequencyData}
\end{table}


\begin{figure}[H]
	\centering
	\includegraphics[width=\textwidth]{images/overall}
	\caption{A comparison of vortex shedding frequency with increasing $n$ from 2 \textendash\ 12.}
	\label{fig:overall} 
\end{figure}





A quadratic line of best fit produced an $R^2$ value of 0.6977, suggesting that the variation is reasonably well represented by a quadratic model. The graph portrays a negative correlation, indicating that bodies with an increased number of streamwise faces produce vortices at a lower frequency \textemdash\ attributable to an overall decrease in Strouhal number with increasing streamwise faces.



\clearpage

\subsection{Results of Practical Investigation}
\label{sec:resultsPractical}
The found average flow velocity was \SI{0.04}{\meter\per\second}. Despite many attempts to achieve laminar flow, the flow remained turbulent, as demonstrated by the irregular wake patterns and a calculated Reynolds number of 800 \textemdash\ determined using \Cref{eq:reynoldsNumber} where $U = \SI{0.04}{\meter\per\second}$, $\nu = \SI{1e6}{\meter\squared\per\second}$, and $L = \SI{0.02}{\meter}$. Nevertheless, the practical experiment yielded qualitative value (see \Cref{lst:snapshots}). 


	



	
	
	
	




