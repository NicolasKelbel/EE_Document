\section{Conclusion}
\label{sec:conclusion}
The goal of this study was to investigate how the number of streamwise faces $n$ of a bluff body influence the vortex shedding frequency in laminar flow. Bluff bodies ranging from $n = 2 \text{ \textendash\ } 12$ were analyzed both theoretically and practically. However, only the theoretical investigation yielded quantitative results.

The research question was thoroughly explored by theoretical means using OpenFOAM simulations, allowing for precise measurements of the fluctuations of the lift coefficient, and therefore an accurate determination of the vortex shedding frequency. Although the practical component of this investigation failed to achieve laminar flow conditions, it provided beneficial visual insight into the mechanism of vortex shedding, enhancing the investigation's tangibility while aiding conceptual understanding.

It was found that there is an overall decrease in vortex shedding frequency as the number of streamwise faces $n$ of a bluff body increases \textemdash\ confirming the previously mentioned hypothesis. This trend is exemplified by bluff body $n = 2$ which produced a vortex shedding frequency of $16.52346\, Hz$, while the bluff body $n = 12$ produced a vortex shedding frequency of $16.21797\, Hz$. The overall trend is in agreement with literature such as \textcite{goncalves1999strouhal} which concluded that a higher number of streamwise faces leads to a decreased Strouhal number and therefore a decreased vortex shedding frequency. However, the effect of bluff body geometry on vortex shedding frequency has only been examined for a limited selection of shapes, leaving countless configurations, including many of those investigated in this study, previously unexplored \parencite[22]{city7565}.

The measured values from $n = 2 \text{ \textendash\ } 7$ follow a fluctuating pattern with periodic increases and decreases. It can be observed that bluff bodies with an even number of streamwise faces $n$ tend to exhibit a greater vortex shedding frequency than those with preceding odd number of faces. Therefore, one may conclude that bluff bodies with an odd number of streamwise faces exhibit an overall lower vortex shedding frequency than ones with an even number of streamwise faces.

Bluff body $n = 8$ seems to be an anomaly with a vortex shedding frequency of $16.32387 Hz$. Although this is lower than the vortex shedding frequency of bluff body $n = 2$, therefore confirming the overall trend, its vortex shedding frequency is higher than that of bluff body $n = 6$. 

Bluff body $n = 9$ exhibits a significantly smaller decrease (only $0,02066\, Hz$) in vortex shedding frequency to bluff body $n = 7$, possibly indicating a shift away from the previous overall downtrend. When considering bluff bodies $n = 9 \text{ \textendash\ } 12$, one identifies a positive correlation between the vortex shedding frequency and the number of streamwise faces $n$, contrary to the previous downtrend. It seems that as the leading edge becomes more circular, the vortex shedding frequency tends to increase. Further investigation would be required to confirm this claim.

A quadratic line of best fit produced an $R^2$ value of 0.70, indicating a moderate non-linear correlation between the number of streamwise faces $n$ and the vortex shedding frequency. The rather low $R^2$ value coupled with the anomaly of bluff body $n = 8$ may be attributed to numerical uncertainties of the simulation \parencite{city7565}. The anomalously low amplitudes of fluctuations of lift coefficient for bluff bodies $n = 7$ and $n = 9$ provide further evidence of possible numerical uncertainties.



