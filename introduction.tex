\section*{Introduction}
Despite its widespread significance, the study of fluid dynamics is largely overlooked by the IB Physics syllabus. Defined by the American Heritage Dictionary as “the branch of applied science that is concerned with the movement of liquids and gases”, fluid dynamics is one of the two subfields of fluid mechanics, “the study of fluids and how forces affect them”.

From the air we breathe to the water we consume; fluids are present in almost every aspect of our lives. Although observing fluids in action is a daily occurrence for all, countless remain unaware of the highly complex and intricate theories governing these seemingly simple and elementary phenomena. Ultimately, a profound youthful interest in cars and planes - everything with an engine really - coupled with a fascination for the motion of fluids led me to the topic of this essay: vortex shedding. Specifically, this essay will concern itself with the question: \textbf{How does increasing the number of faces of a bluff body (ranging from 2 to 12) affect the vortex shedding frequency in laminar flow, measured in Hz?}

Over the past century, vortex shedding has garnered a multifold of attention, with hundreds of papers published, allowing one to consider it as one of the most extensively studied phenomena in fluid mechanics of this time. This partially elucidated phenomenon is quintessential in a broad range of scientific and engineering contexts. Bridges may suffer from vortex shedding excitation, a severe challenge which undermines their structural integrity. This was exemplified by the collapse of the Tacoma Narrows Bridge on the 7th of November 1940, where vortex shedding acted as the instigating factor. In a nutshell, the natural frequency of the bridge’s vertical vibration was approximately congruent to the vortex shedding frequency measured over a stationary deck (the vortex shedding frequency measured when the bridge deck is not vibrating), leading to resonance, amplifying the amplitude of the vortex shedding induced oscillation of the bridge deck dramatically. The near identical frequencies, also referred to as the lock-in effect, led to torsional vibrations, promoting the bridge’s collapse. Although the failure of the Tacoma Narrows Bridge was attributed to torsional vibrations, the lock-in effect was the initiating factor, therefore holding utmost significance. Aside from maintaining bridges, vortex shedding will continue to hold significance, especially when considering the rapid progress in the field of aerospace.

As previously mentioned, the IB Physics Syllabus doesn’t directly concern itself with vortex shedding or, in general, the majority of fluid dynamics. Nevertheless, links can be made to section C: Wave Behavior. Section C.1, Simple Harmonic Motion, concerns itself with, as the name suggests, simple harmonic motion, “the simplified theoretical model representing oscillations”. Basic concepts such as frequency and time period are discussed,  ideas required in order to analyze the frequency of the vortex shedding caused by a bluff body Furthermore, the topic of this essay also links to section C.4, Standing Waves and Resonance. Concepts of natural frequency, vibrations and resonance are key when discussing the problems caused by vortex shedding and the significance of the frequency of vortex shedding, like the one detailed above. 