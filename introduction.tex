\section*{Introduction}
Despite its widespread significance, the study of fluid dynamics is largely overlooked by the IB Physics syllabus. Defined by the American Heritage Dictionary as “the branch of applied science that is concerned with the movement of liquids and gases”, fluid dynamics is one of the two subfields of fluid mechanics, “the study of fluids and how forces affect them”. 

From the air we breathe to the water we consume; fluids are present in almost every aspect of our lives. Although observing fluids in action is a daily occurrence for all, countless remain unaware of the highly complex and intricate theories governing these seemingly simple and elementary phenomena. Ultimately, a profound youthful interest in cars and planes - everything with an engine really - coupled with a fascination for the motion of water lead me to the topic of this essay: vortex shedding. Specifically, this essay will concern itself with the interplay between a bluff body and the frequency of vortex shedding it causes in laminar flow. 

Vortex shedding has gained a multifold of attention, with hundreds of papers being published in the last century, making it one of the most studied phenomena of fluid mechanics. This partially elucidated phenomenon is quintessential in a broad range of scientific and engineering contexts. When considering bridges,