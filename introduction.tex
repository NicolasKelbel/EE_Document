\section{Introduction}
From the air we breathe to the water we consume; fluids are present in almost every aspect of our lives. Although observing fluids in action is a daily occurrence for all, countless remain unaware of the highly complex and intricate theories governing these seemingly simple and elementary phenomena. Ultimately, a profound youthful interest in cars and planes \textemdash\ everything with an engine really \textemdash\ coupled with a fascination for the motion of fluids led to the topic of this essay: vortex shedding. Specifically, this essay will concern itself with the question: \textit{How does increasing the number of streamwise faces of a bluff body (ranging from 2 to 12) affect the vortex shedding frequency in laminar flow, measured in \si{\hertz}, in a two-dimensional plane?}

Over the past century, vortex shedding has garnered a multifold of attention, with hundreds of papers published \parencite[61]{buresti1998}. This partially elucidated phenomenon is quintessential in a broad range of scientific and engineering contexts, from maintaining ubiquitous infrastructure to developing cutting-edge aerospace technologies. Bridges may suffer from vortex shedding excitation, a severe challenge which undermines their structural integrity \parencite[1040]{jurado2012}. This was exemplified by the collapse of the Tacoma Narrows Bridge on the 7th of November 1940, where vortex shedding acted as the instigating mechanism -- triggering vertical vibration that eventually transitioned into the torsional oscillations, ultimately causing its collapse \parencite{tacoma_bridge_vibrations}.

The IB Physics Guide \parencite{ib_physics_2025} does not directly concern itself with vortex shedding or, in general, the majority of fluid dynamics. Nevertheless, links can be made to Section C: \textit{Wave Behavior}. Section C.1, \textit{Simple Harmonic Motion}, concerns itself with an idealized theoretical representation of the behavior of oscillatory systems \parencite[313]{allum2023}. Fundamental concepts such as frequency and time period are discussed, ideas required in order to analyze the frequency of the vortex shedding caused by a bluff body. Furthermore, the topic of this essay also links to Section C.4, \textit{Standing Waves and Resonance}. Concepts of natural frequency, vibrations and resonance are key when discussing the problems caused by vortex shedding.





