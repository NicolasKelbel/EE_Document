\section{Introduction}
Over the past century, vortex shedding has garnered a multifold of attention, with hundreds of papers published \parencite[61]{buresti1998}. This partially elucidated phenomenon is quintessential in a broad range of scientific and engineering contexts, from maintaining ubiquitous infrastructure to developing cutting-edge aerospace technologies. Bridges may suffer from vortex shedding excitation, a severe challenge which undermines their structural integrity \parencite[1040]{jurado2012}, exemplified by the 1940 collapse of the Tacoma Narrows Bridge \parencite{tacoma_bridge_vibrations}.

Together with a profound youthful interest for aviation, this intrigue ultimately inspired the topic of this essay. Specifically, this investigation will concern itself with the question: \textbf{How does increasing the number of streamwise faces of a bluff body (ranging from 2 to 12) affect the vortex shedding frequency in laminar flow, measured in \si{\hertz}, in a two-dimensional plane?}. The investigation will employ both practical and theoretical methods, the latter serving as a definitive and reliable source of data for analysis.

Links can be made to Section C of the IB Physics Guide: \textit{Wave Behavior} \parencite{ib_physics_2025}. Section C.1, \textit{Simple Harmonic Motion}, concerns itself with an idealized theoretical representation of the behavior of oscillatory systems \parencite[313]{allum2023}. Fundamental concepts such as frequency and time period are discussed, ideas required in order to analyze the frequency of the vortex shedding caused by a bluff body. Furthermore, the topic of this essay also links to Section C.4, \textit{Standing Waves and Resonance}. Concepts of natural frequency, vibrations and resonance are key when discussing the problems caused by vortex shedding.









